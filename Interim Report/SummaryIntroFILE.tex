
\section{Summary} 
\textcolor{red}{$\Box$ write this after body is done}\\What we did, what’s left to do, how we intend to approach the remaining work\\
\\\textcolor{red}{$\Box$ ok to take from proposal verbatim for the intro?   For now, Section II. Intro is lifted nearly verbatim from Proposal}
\section{Introduction}
Enabling \SSPARC systems is an important objective that will mitigate spectrum congestion.  Solutions to solve this objective will require dynamic allocation of time/frequency/space slots among all spectral users.  Regardless of the allocation approach, precise signal containment in both spectral and spatial dimensions is required.  After all, without the ability to limit a signal to a desired spatial/spectral window, allocation optimization becomes irrelevant.  \textcolor{red}{$\Box$ Define SSAS here?}\par


\subsection{SSPARC \textcolor{red}{\fbox{?}delete this subheading? should be SSAS not SSPARC?}}

This work contains several innovations that can be decomposed into two categories: i) techniques to reduce the errantly transmitted energy due to nonlinear distortions in \SSPARC radios and radars, and ii) techniques that operate on the receiver side to mitigate nonlinearly distorted signals.  All of the proposed innovations aim to increase concurrent operating capability of radar and communication systems.  The proposed innovations can be applied to either codesign or coexistence \SSPARC systems.  Furthermore all of these innovations can be agnostically paired to other \SSPARC mechanisms that are developed by other\line(1,0){50}  \textcolor{red}{$\Box$ insert term for SSPARC participants ... was ``bidders.''  And introduce SSAS before next heading because abbrevs is not working in headings}
\subsection{SSAS Problem Discussion
 (Pull material from proposal; outline all three thrusts)\textcolor{red}{\fbox{?}delete this subheading?}}
\subsubsection{Thrust 1. Algorithms to mitigate spectral artifacts for multiplexed signals}
It is well-known that signals with high dynamic range tend to be distorted by system nonlinearities.  It is possible to have low amounts of distortion even for signals with high dynamic range, but this comes at the expense of low power efficiency.  That is, reducing dynamic range can improve the operating point of a system in the distortion-power-efficiency trade-off space. Ideally, signals are designed to either be constant modulus (CM) or have a low peak-to-average power ratio (PAR).

Shaping radar signals to be CM is relatively easy since radar signals do not carry information and are not multiplexed with other signals.  However, when other signals are multiplexed or when information is encoded in the signaling, the problem is much more complicated.  Under the SSPARC paradigm, nodes may transmit multiplexed signal aggregation that contain both communications signals and radar signals. 
While multiplexing both deterministic signaling (radar) and information-bearing signaling (communications) through the same radio frequency (RF) transmit chain will necessarily increase the signal PAR on average and thus increase nonlinear distortions, signal multiplexing also provides additional degrees of freedom for optimizing the joint signal.  Our innovation is to leverage our past experience in minimizing nonlinear distortion for communications signals to solve this new problem of minimizing the distortion for radio/radar multiplexed signals.  Our solution will be applicable for both the codesign and the coexistence paradigms.  
It is well-known that signals with high dynamic range tend to be distorted by system nonlinearities.  It is possible to have low amounts of distortion even for signals with high dynamic range, but this comes at the expense of low power efficiency.  That is, reducing dynamic range can improve the operating point of a system in the distortion-power-efficiency trade-off space. Ideally, signals are designed to either be \CM or have a low \PAR.
Shaping radar signals to be \CM is relatively easy since radar signals do not carry information and are not multiplexed with other signals.  However, when other signals are multiplexed or when information is encoded in the signaling, the problem is much more complicated.  Under the \SSPARC paradigm, nodes may transmit multiplexed signal aggregation that contain both communications signals and radar signals. 
While multiplexing both deterministic signaling (radar) and information-bearing signaling (communications) through the same \RF transmit chain will necessarily increase the signal PAR on average and thus increase nonlinear distortions, signal multiplexing also provides additional degrees of freedom for optimizing the joint signal.  Our innovation is to leverage our past experience in minimizing nonlinear distortion for communications signals to solve this new problem of minimizing the distortion for radio/radar multiplexed signals.  Our solution will be applicable for both the codesign and the coexistence paradigms.  

\subsubsection{Thrust 2.  Algorithms to mitigate spatial artifacts for spatially shaped signals}
In multi-antenna systems signals are sent to each antenna element and then weighted and/or “phased” to achieve a desired aggregate beam pattern.  When there is unmitigated nonlinear distortion in the RF chain, due to the power amplifier, for instance, the beam pattern of the transmission will also be distorted.  We propose two innovations for correcting this problem:
\renewcommand{\theenumi}{\roman{enumi}}%
\begin{enumerate}
\item Optimizing beamforming weights with nonlinear Tx chains: We will optimize the transmitting beam pattern by taking the nonlinear distortion into account when determining the beam weights.  
\item Minimizing beam error through signal modification: A more holistic approach is to jointly optimize signal distortion and the antenna weights.  This will provide better performance, but may not be possible in certain systems where the signal processing for the transmit signaling is separated from the beam-weight optimization signal processing.  When joint optimization is possible, the PAR of the transmitting signal is jointly optimized with the beam weights.  Under this paradigm, we can also optimize the signal to minimize spectral splatter, thereby performing the spectral and spatial optimizations in a coupled way through a single globally optimization procedure.
\end{enumerate}

\subsubsection{Thrust 3.  Receiver-side mitigation of spectral artifact noise}
The nonlinear distortion caused by RF chains exhibits specific qualities defined by the nonlinear distortion function.  For example, the distortion from clipping nonlinearities tends to have a non-Gaussian and non-white distribution.  Our innovation is to use this information at the receive side of radars/radios to improve the system performance.  For instance, communications signals are drawn from a finite signal constellation.  With knowledge of the constellation, as well as partial information about the nonlinearities, we can use decision-directed techniques to subtract distorted communications inference \textcolor{red}{[interference?  so it's a radar signal of interest?  see question list - are we doing this?]} from a signal of interest.  \textcolor{green}{[the following is more closely related to what we've done -- beef it up?  keep previous section?]}
The low-entropy signals typically used as radar sounding pulses are also amenable to being subtracted from desired signals. This can be more complicated or impossible if the radar signal saturates the receiver front end.  However, often, high-power radars will be given a large guard band so that users are separated by large spectral distances.  In this situation, the interference to neighboring users is purely nonlinear distortion of the radar signal.  We are proposing to use decision-directed techniques coupled with noise whitening techniques to mitigate this kind of adjacent channel interference.
